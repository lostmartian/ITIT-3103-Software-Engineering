% Document by Sahil Gangurde %
\documentclass[12pt]{article}

\usepackage{amssymb,amsmath,amsfonts,latexsym,graphicx}


\setlength{\oddsidemargin}{0.2cm}
\setlength{\textwidth}{16.5cm}
\setlength{\topmargin}{-2.0cm}
\setlength{\textheight}{24.5cm}   			

\pagestyle{empty}


\begin{document}

\begin{center}

\textbf{Indian Institute of Information Technology, Gwalior \\ Software Engineering(ITIT-3103)}

\vspace{0.5cm}

\textbf{Project Title}: Semi-anonymous image based message board website
\vspace{0.5cm}

\textbf{Group Members(Grp No. 24)}: Anish Jaiswal (2019IMT-017), Sahil Gangurde (2019IMT-034) and Shashi Kumar (2019IMT-090) 

\end{center}

\section{Project Proposal}

\subsection{Problem to be solved}
Since the past several years increasing distrust and spread of misinformation has been happening online over various social media platforms. This has lead to create unhealthy atmosphere over the social media pages and has imposed rules and regulations on customers which weren't present some years ago. To have a \textbf{healthy, unrestricted, uncensored and fully democratic discussion and posting of content} from topics of different genres such as sports, gaming, memes, etc we propose to create such a online web application.

\subsection{Importance of such a platform}
In today's era with increasing interference of administration of big giants(countries, companies and authorities) meaningful conversations/discussions/content not in favour organizations are removed or unnecessarily flagged. At such times a \textbf{social media discussion and content posting platform} completely isolated from regulations, owned and managed by people completely comes in handy as a general necessity.

\subsection{Customer Overview}
A customer 'C' who wants to have a personal opinion which is quite unpopular from the rest or one who want to raise an important issue or one who want to share some incident anonymously and see the reactions of others over that issue or content will like to use this system due to unbiased/anonymous opinion or vote. Also 'C' would like to have a simple user friendly UI which will be out one of the top priorities.

\subsection{Difficulty curve}
This project is a perfect fit for our group. We are good with recent frontend web framework/libraries technologies like React and/or Angular. Over the course of this semester we will be learning along the way backend technology and implement basic to advanced modules such as database connectivity, user authentication and CRUD operations. As we will be learning backend on the go and one semester time is enough to implement the above specifications we conclude that \textbf{this project neither ambitious nor a cakewalk for us}.

\section{Feasibility Study}

\subsection{Existing Systems and platforms}
\textbf{4chan} and all the other websites inspired by it work on the same principle serving as a discussion platform without interference of the intermediaries. We are attempting to implement a low-level version of this platform with added benefit of \textbf{semi-anonymous and brownie points system} while keeping the motive alive. 4chan lacks important feature of having a good UI which we will be improving on. Also complete anonymity is not a cool thing if you want certain posts to be known by your real self so we will be implementing user authentication as well as anonymous posting ways of handling posts.

\subsection{Scope and Deliverables}
At the end we try to provide a fully built website having:
\begin{itemize}
    \item Secure user registration + Anonymity implementation
    \item CRUD operations on user posts
    \item Voting system for posts
    \item Brownie points for registered users based upvotes/downvotes on their posts created
\end{itemize}
In the given time frame(1 semester) we do not plan to deliver/invest time on modules like: 
\begin{itemize}
    \item Secure storage
    \item Bot verification
    \item User interaction such as direct messaging
    \item Architectural issues such as scalability
\end{itemize}
Though we initially plan not to deliver it but if time permits and our understanding of the techstack grows we may implement the above non-deliverable features.

\subsection{Process to be followed and plan outline}
We will be using the \textbf{iterative waterfall model} as the requirements are well defined and also the some part of techstack is new to us. The outline plan would be first we will \textbf{design the backend systems and draw relations with various entities present and also relations between multiple modules}. Once the backend is designed we will move to the frontend designing and once the designing phase is over we will then start to code the actual project and then test and improve upon.

\subsection{Financial feasibility}
The techstack and the resources needed to learn/implement those are available for \textbf{free of cost} so there is no need for any financial funding as such.

\subsection{Technical Feasibility}
The techstack we will be using mainly comprises of \textbf{React.Js(frontend) and Spring Boot (backend)}. The technologies mentioned above can work with any low hardware personal computers which our team already possess. Our team is highly qualified in the mentioned technologies and given the time constraint for completion the project can be implemented easily and can be delivered. Also the techstack can expand as per need if we try to implement more features and according to the current latest market standards.

\subsection{Resource and Performance Analysis}
The resources required to create and run the application are \textbf{negligible in terms of cost, size and time}. Any modern day browser can run the application created. To create the application modern day processing power in a normal laptop is enough and is available to us. Also things such as programming tools, hosting solution etc can be done with free services/university aided programs which are easy to use and also fast. 

\subsection{Risks involved}

As far as the technical risks go, scalability for such a website is an important factor. As the user base increases a need for distributed service can be felt. As talked in Scope and Deliverables scalability is not something we are concerned about right now so there is little to no risk involved in the technical field.

Such types of platforms are never been supported by government organizations, public agencies and/or some ideological societies. This leads to shut down/ban of such services but the people who believe in free, raw and unbiased democratic internet support this cause. This type of platform \textbf{isn't aimed to formulate vague, absurd ideologies or to spread hate speech against anyone}.

\subsection{Social/Legal Issues}
As mentioned above this type of platform may come under radar in some countries/locations where the law is very strict regarding internet content posting. If such an issue occurs where the post really can cause any social distress(in a bad way) then the administration of the website can remove the post. Apart form this there are \textbf{no legal issues} with this platform whatsoever.

\end{document}